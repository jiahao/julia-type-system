\begin{abstract}
Programming in applied math and the sciences has a unique set of challenges. As a result, popular languages tend to suffer from issues of programming and performance and ease of reasoning, while more principled solutions based on programming language theory suffer from lack of adoption. A major issue is that while static types improve performance and ease of reasoning, they are often ill-suited for the experimentation required for scientific computing. Unlike many programs computer scientists write, scientific programs often have inputs of uncertain type that make it difficult to construct a statically typed program.

We have designed the type system of the Julia programming language to address the problems of performance and reasoning in scientific programming. Julia supports multiple dispatch over dynamic parametric types. The key insight in the design of the type system is that it supports \emph{type tags}, optional static annotations that can be used for both performance and reasoning. \TODO{Have some sort of static reasoning argument too. It's okay if it's not implemented.} We describe the algorithm for taking advantage of these static annotations provides \TODO{how many x} speedup over code without annotations in a set of representative benchmarks. In addition, we describe how the Julia standard library, used by \TODO{how many} users, takes advantage of Julia's flexible dispatch mechanism for extension and reuse.
\end{abstract}
