\begin{abstract}
Programmers in the sciences and applied math often prefer dynamically typed languages. These domains often benefit from good performance, as many of these programs need to process large amounts of data. Unfortunately, dynamic languages often suffer from poor performance. We have designed the Julia programming language based on the hypothesis that dynamic method dispatch is a major bottleneck in scientific programs. To address this problem, Julia supports efficient multiple dispatch over dynamic parametric types. The key insight in the design of the type system is that it supports \emph{type tags}, optional static annotations that can be used to determine function dispatch. We describe the algorithm for taking advantage of these static annotations provides \TODO{how many x} speedup over code without annotations in a set of representative benchmarks. In addition, we describe how the Julia standard library, used by \TODO{how many} users, takes advantage of Julia's flexible dispatch mechanism for extension and reuse.
\end{abstract}
