\begin{abstract}
Technical computing has a split personality. On one extreme, it is associated with big computations,
supercomputers, and high performance. On the other, it is associated with high-level tools for domain
expert productivity that contain rich, polymorphic mathematical operations.
%For performance reasons, scientific languages often have type-level distinctions. However, users
%often want to reuse the code and data types in ways that are not statically analyzable. 
A popular solution to reconciling these extremes is to have two-language environment: a low-level
statically-typed language like C, C++ or Fortran for performance, and a high-level language such as
MATLAB, Python/NumPy or R, for dynamic dispatch.
Two-language approaches are unwieldy, requiring static and dynamic components to be separated by hand.
The language barrier is also inflexible, inhibiting optimization between user and library code.

As a solution to the two-language problem, we have designed the Julia programming language to be a
high-level technical language with good performance. The key innovation is the design of a type
system that supports a combination of dynamic type-based dispatch and static data flow type inference.
We describe the type system and provide examples demonstrating how key
language features aid in the implementation of real-world linear algebra libraries.
We describe how Julia yields comparable performance on standard benchmarks compared to technical computing systems written in C, C++, and Fortran, 
as well as how users take advantage of the flexibility of Julia's dynamic dispatch.
\end{abstract}
