\begin{abstract}
Technical computing has a split personality: it is associated with big
computations, supercomputers, and high performance, but also with high-level
tools for domain expert productivity that contain rich, polymorphic
mathematical operations and multidimensional arrays. In this context,
type-level distinctions are introduced for performance. However users often
want to reuse the same code and data types in ways that are not statically
analyzable. One popular solution is to have a two language environment: a
low-level statically-typed language like C, C++ or Fortran for performance, and
a high-level language such as MATLAB, Python/NumPy or R, for dynamic dispatch.
However, two-language approaches are unwieldy, requiring static and dynamic
components to be separated by hand. The language barrier also inhibits
optimization between user and library code.

Julia is a programming language designed to address the two-language problem.
We show here the combination of dynamic type-based dispatch and data flow type
inference allows good performance to be obtained from high-level technical
programs. We provides examples demonstrating how key language features aid in
the implementation of various real-world linear algebra codes.
\end{abstract}
