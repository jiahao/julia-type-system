\section{Motivating Example}

\subsection{Why is dynamic multiple dispatch useful for numerical computing?}

Dynamic multiple dispatch is valuable for writing generic numeric code which
handles the broadest possible set of inputs while at the same time making use
of handcrafted performant kernels when they are available. An illustration of
this principle can be seen in Julia's linear algebra routines. Many systems
provide optimized implementations of standard libraries such as BLAS (Basic
Linear Algebra Subprograms) and LAPACK (Linear Algebra PACKage) which provide
standard functionality such as matrix-matrix multiplication and spectral
decompositions for computing eigenvalues and eigenvectors. Implementations such
as ATLAS, GotoBLAS and OpenBLAS provide performant kernels customized for
individual microarchitectures that can reach a significant fraction of
theoretical peak FLOPS.

There are a few limitations, however. Chief among these is the cognitive burden
for users to learn how these routines are named and figuring out which routine
performs the desired computations. These libraries, like many scientific codes,
are written in Fortran, with no notion of polymorphic dispatch. Most of these
routines have six letter names in line with traditional Fortran 77 subroutine
nomenclature.

\begin{enumerate}

\item The first letter of each routine name describes the floating point type
(e.g.\ single precision real, double precision real, single precision complex
and double precision complex).

\item The next two letters is a code for the type of matrix (e.g.\ \code{GE} for
general, \code{ST} for symmetric tridiagonal).

\item The last few letters describe the type of computation to carry out (e.g.\
\code{SV} for linear solver, \code{EIN} for eigenvalues and eigenvalues). This
produces a proliferation of routines with a large amount of redundant code.
Furthermore, it is very difficult to extend these libraries with new
functionality, as many redundant routines have to be written even something as
fundamental as adding a new level of floating-point precision.

\end{enumerate}

In contrast, Julia's system of generic functions with polymorphic dispatch
system allows a compact factorization of the namespace of linear algebra
routines. Instead of a routine name like \code{DSTEIN} (double precision
symmetric tridiagonal eigenproblem solver), the \code{LinAlg} module in base
Julia introduces the function \code{eig} which solves eigensystems, and allows
parametric method signatures like
\code{eig\{T<:BlasFloat\}(M::SymTridiagonal\{T\})} which allow specialized
methods to be defined for both the matrix type (\code{SymTridiagonal},
symmetric tridiagonal) and its element type (\code{BlasFloat} is a union type
\code{Union(Float32, Float64, Complex\{Float32\}, Complex\{Float64\})}, and so
\code{T<:BlasFloat} specifies a template that defines four separate,
specialized methods. The \code{LinAlg.LAPACK} module also uses Julia's
metaprogramming capabilities to map Julia function calls onto their LAPACK
equivalents. In this way, multiple dispatch allows a handful of overloaded
functions to accept the widest possible set of inputs while not giving up the
ability to use handcrafted performant kernels when they are available.

The ability to add new specialized methods to existing generic functions allows
new functionality to be added seamlessly. For example, Julia provides generic
fallback routines to do matrix computations over arbitrary fields of element
types \code{T}, providing the ability to compute on numeric types which are not
mappable to hardware floating point types. This can be useful to perform matrix
computations in exact rational arithmetic or software--emulated higher
precision floating point arithmetic to verify the implementations of algorithms
or to detect the possibility of numerical instability associated with roundoff
errors. Oftentimes it is possible to write a single kernel that will work
independently of the underlying field \code{T}, avoiding the need to write
redundant code.

The ability to perform multiple dispatch dynamically is valuable when you can
specialize on certain algorithms but you can only determine which algorithm to
dispatch on at runtime. This is useful for linear algebraic applications where
certain properties of an input matrix are difficult to determine \textit{a
priori}, such as matrix symmetries like Hermiticity, or spectral properties
like positive definiteness. For example the matrix square root function
(\code{sqrtm}) includes a runtime check to detect if the input matrix is
symmetric (for real element types) or Hermitian (for complex element types). If
the matrix is symmetric/Hermitian, \code{sqrtm} dispatches on a faster,
specialized method to compute the principal matrix square root using the
eigenvalues and eigenvectors, otherwise a different method is dispatched upon
which computes and uses the Schur factorization instead.

The generality afforded by dynamic dispatch allows for runtime dispatch onto
optimized kernels depending on the input matrix. For example, the
\code{sqrtm\{M::Matrix\}} function for computing matrix square roots contains a
runtime check to detect if the input matrix is symmetric. If it is, then the
computation is dispatched onto a \code{sqrtm\{S::Symmetric\}} method which
computes more efficiently than the general case. Of course if the user knows at
the time of writing the code he or she could call the specialized method
directly, but the ability to do dynamic dispatch allows the same performant
kernel to be called even in situations where the user does not know about the
special symmetry of a particular input matrix, or perhaps may not even be aware
of specialized algorithms that can be enabled by such information that can be
deduced at runtime.

The \code{factorize} function in Julia is another, more extreme example of how
dynamic multiple dispatch can be used to detect matrix symmetries and structure
(e.g.\ if a matrix is diagonal, symmetric tridiagonal, upper triangular, etc.)
and compute appropriate matrix factorizations that best take advantage of the
detected matrix properties. The output is a matrix factorization object whose
specific type is only defined at runtime depending on the specific matrix
properties that are detected at runtime. For example, calling \code{factorize}
on a symmetric positive definite matrix returns a \code{Cholesky} type
representing a Cholesky factorization, whereas a symmetric indefinite matrix
would yield a \code{BunchKaufman} type encoding a Bunch--Kaufman factorization
instead. Positive definiteness is a good example of a matrix property that can
be difficult to determine \textit{a priori} and the fastest way to detect it
would be to attempt to compute a matrix factorization that succeeds only on
postiive definite matrices (such as the Cholesky decomposition), and fallback
to returning a different type if the computation failed.

Generic functions express a common action or computation, which may in
practice be performed in many different ways depending on the inputs. For
example, one may wish to compute the mean of a given statistical distribution.
In Julia, the computation can be expressed simply as dispatch onto the method
\code{mean(D::Distribution)}, with a different method for \code{mean} being
defined for different \code{Distribution}s. In contrast, much scientific code
is written in languages that lack generic functions, resulting in users having
to keep track of a large family of very closely related functions which
together can be thought of as specialized mthods for a polymorphic function.
Table~\ref{statsfunctions} shows a family of closely related functions in the R
programming language and their Julia counterparts in the
\package{Distributions.jl} package. Julia provides a common functional interface
for each type of statistical distribution, with generic functions like
\code{pdf} to compute the probability density function, \code{cdf} for the
cumulative density function, \code{quantile} to compute quantiles, and
\code{rand} for sampling random variates. In contrast, each distribution in R
must have its own associated family of functions prefixed by \code{d},
\code{p}, \code{q} and \code{r} respectively, and the number of functions grows
by both the number of distributions and the number of desired computations on
them.

\code{pdf}
\begin{table}
{\tiny
\begin{tabular}{l || l | l | l | l}
  \textbf{\code{D}}                & \textbf{\code{pdf(D)}}    & \textbf{\code{cdf(D)}}    &
        \textbf{\code{quantile(D)}} & \textbf{\code{rand(D)}}  \\ \hline\hline
  \textbf{\code{Beta}}             & \code{dbeta}     & \code{pbeta}     & \code{qbeta}       & \code{rbeta}    \\
  \textbf{\code{Binomial}}         & \code{dbinom}    & \code{pbinom}    & \code{qbinom}      & \code{rbinom}   \\
  \textbf{\code{Cauchy}}           & \code{dcauchy}   & \code{pcauchy}   & \code{qcauchy}     & \code{rcauchy}  \\
  \textbf{\code{Chisq}}            & \code{dchisq}    & \code{pchisq}    & \code{qchisq}      & \code{rchisq}   \\
  \textbf{\code{Exponential}}      & \code{dexp}      & \code{pexp}      & \code{qexp}        & \code{rexp}     \\
  \textbf{\code{Fdist}}            & \code{df}        & \code{pf}        & \code{qf}          & \code{rf}       \\
  \textbf{\code{Gamma}}            & \code{dgamma}    & \code{pgamma}    & \code{qgamma}      & \code{rgamma}   \\
  \textbf{\code{Geometric}}        & \code{dgeom}     & \code{pgeom}     & \code{qgeom}       & \code{rgeom}    \\
  \textbf{\code{Hypergeometric}}   & \code{dhyper}    & \code{phyper}    & \code{qhyper}      & \code{rhyper}   \\
  \textbf{\code{LogNormal}}        & \code{dlnorm}    & \code{plnorm}    & \code{qlnorm}      & \code{rlnorm}   \\
  \textbf{\code{Multinomial}}      & \code{dmultinom} & \code{pmultinom} & \code{qmultinom}   & \code{rmultinom}\\
  \textbf{\code{NegativeBinomial}} & \code{dnbinom}   & \code{pnbinom}   & \code{qnbinom}     & \code{rnbinom}  \\
  \textbf{\code{Normal}}           & \code{dnorm}     & \code{pnorm}     & \code{qnorm}       & \code{rnorm}    \\
  \textbf{\code{Poisson}}          & \code{dpois}     & \code{ppois}     & \code{qpois}       & \code{rpois}    \\
  \textbf{\code{TDist}}            & \code{dt}        & \code{pt}        & \code{qt}          & \code{rt}       \\
  \textbf{\code{Uniform}}          & \code{dunif}     & \code{punif}     & \code{qunif}       & \code{runif}    \\
  \textbf{\code{Weibull}}          & \code{dweibull}  & \code{pweibull}  & \code{qweibull}    & \code{rweibull} \\
\end{tabular}
}
\label{statsfunctions}
\caption{List of common distributions in R v3.1.1\cite{rlang} and their
associated functions. Shown in bold are the equivalent Julia generic
functions and \code{Distribution} types provided by the Julia package
\package{Distributions.jl}, showing how the namespace
is simplified by generic functions and multiple dispatch.}
\end{table}

\TODO{show code examples}

\TODO{Other possible examples from Distributions.jl: QQ plots, posterior function}

\subsection{External dispatch with dynamic multimethods}

One thing you can do with dynamic multiple dispatch is external dispatch,
i.e.\ to be able to add new methods to an existing type rather than
specifying all the possible methods upfront. This is used to good effect
in Julia packages to extend functionality of the base library. For example,
the package \package{Color.jl} for manipulating color spaces extends the
\code{writemime} base function, which writes an object as a specified
MIME type\cite{mimerfc} to an I/O stream. \code{Color.jl} extends
\code{writemime} with new methods that draw color swatches.

\TODO{Here, we could put a figure showing writemime being called on a base
Julia object, writemime(::Color), and writemime(::VectorColor) objects.
Possibly take a screenshot in IJulia.}

External dispatch also facilitates interaction between Julia packages---
colors can be rendered using the \package{Cairo.jl} package,
which allows rendering to a Cairo backend~\cite{cairographics}.
When using Cairo to render an object in a given color, \code{Cairo.jl}
doesn't need to know anything about the underlying color space; it just
needs to call \code{convert} methods provided by \code{Color.jl}.
Julia's \code{convert} protocol is the ``narrow waist'' that allows
packages to allow using all kinds of color spaces naturally, without
needing to know anything specifically about them.

External dispatch can be performed in other languages like C++. However,
C++ only allows static external dispatch, which severely limits its utility.
Furthermore, external dispatch in C++ requires virtual methods and visitor
patterns\cite{designpatterns} to work around the absence of explicit support
for external dispatch.
