\section{Introduction}

\paragraph{Limitations of class-based dispatch for numerics} %TODO These paragraph markings are structural only; take them out later.

Argurably, class-based dispatch is not natural for numeric computations. It's difficult to know upfront all the possible things you want to do to a floating point number.

In languages like C++, you can extend existing classes with new methods, but it's difficult. Need to have virtual methods and invoke template specialization with them. Oftentimes you also need runtime code to decide what kind of object to create.

Static languages like Haskell have typeclasses, but you'll need to anticipate all the necessary use cases ahead of time because everything is resolved statically and also know all the cases you'll be in at compile time. Dynamic languages provide only dynamic bindings,  which means that programmers don't have to reason about the differences between static and dynamic semantics, and dynamic binding is more general anyway.
