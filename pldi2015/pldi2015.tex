%-----------------------------------------------------------------------------
%
%               Template for sigplanconf LaTeX Class
%
% Name:         sigplanconf-template.tex
%
% Purpose:      A template for sigplanconf.cls, which is a LaTeX 2e class
%               file for SIGPLAN conference proceedings.
%
% Author:       Paul C. Anagnostopoulos
%               Windfall Software
%               978 371-2316
%               paul@windfall.com
%
% Created:      15 February 2005
%
%-----------------------------------------------------------------------------


\documentclass[9pt,cm,preprint]{sigplanconf}

% The following \documentclass options may be useful:
%
% 10pt          To set in 10-point type instead of 9-point.
% 11pt          To set in 11-point type instead of 9-point.
% authoryear    To obtain author/year citation style instead of numeric.

\usepackage{pslatex}
\usepackage{todonotes} %TODO Remove for final submission
\usepackage{minted}

\newcommand{\TODO}[1]{[\textsl{#1}]}
\newcommand{\code}[1]{\texttt{#1}}
\newcommand{\package}[1]{\code{#1}\cite{#1}}


\begin{document}

\conferenceinfo{PLDI '15}{June 12---20, 2015, Portland, OR, USA.}
\copyrightyear{2015}
%\copyrightdata{}

%\titlebanner{banner above paper title}        % These are ignored unless
\preprintfooter{DRAFT - do not distribute}   % 'preprint' option specified.

\title{Efficient Multiple Dispatch in Julia}
%\subtitle{Subtitle Text, if any}

\authorinfo{Jeff Bezanson \and Jiahao Chen \and Stefan Karpinski \and Jean Yang}
	{Massachusetts Institute of Technology}
	{bezanson@mit.edu, jiahao@mit.edu, stefan@karpinski.org, jeanyang@mit.edu}

\maketitle

\begin{abstract}
Programmers in the sciences and applied math often prefer dynamically typed languages. These domains often benefit from good performance, as many of these programs need to process large amounts of data. Unfortunately, dynamic languages often suffer from poor performance. We have designed the Julia programming language based on the hypothesis that dynamic method dispatch is a major bottleneck in scientific programs. To address this problem, Julia supports efficient multiple dispatch over dynamic parametric types. The key insight in the design of the type system is that it supports \emph{type tags}, optional static annotations that can be used to determine function dispatch. We describe the algorithm for taking advantage of these static annotations provides \TODO{how many x} speedup over code without annotations in a set of representative benchmarks. In addition, we describe how the Julia standard library, used by \TODO{how many} users, takes advantage of Julia's flexible dispatch mechanism for extension and reuse.
\end{abstract}


\category{D.3.3}{PROGRAMMING LANGUAGES}{Language Constructs and Features}

\terms Languages, Multiple dispatch, Multimethods

%\keywords
%Language design, run-time system

\section{Introduction}

\paragraph{Limitations of class-based dispatch for numerics} %TODO These paragraph markings are structural only; take them out later.

Arguably, class-based dispatch is not natural for numeric computations. It's difficult to know upfront all the possible things you want to do to a floating point number.

In languages like C++, you can extend existing classes with new methods, but it's difficult. Need to have virtual methods and invoke template specialization with them. Oftentimes you also need runtime code to decide what kind of object to create.

Static languages like Haskell have typeclasses\cite{typeclass}, but you'll need to anticipate all the necessary use cases ahead of time because everything is resolved statically and also know all the cases you'll be in at compile time. Dynamic languages provide only dynamic bindings,  which means that programmers don't have to reason about the differences between static and dynamic semantics, and dynamic binding is more general anyway.

\paragraph{Dynamic languages, realism and empiricism}

\begin{quote}
Engineers build things; scientists describe reality; philosophers get lost in broad daylight.---\cite{Gabriel2012}
\end{quote}

Arguably, dynamic languages are more in concord with the empirical mode of scientific inquiry that is familiar to users of technical computing, many of whom are also scientists and engineers. Dynamic languages embody a realist philosophy: programs are not checked for correctness, but are executed until termination or when a runtime error is thrown. The focus is to make sense out of whatever program a user may write. In contrast, static languages focus on formal correctness, validating programming on the basis of satisfying constraints imposed by static analyses. Furthermore, these formal systems tend to concern themselves with only the interface to data types, not with their internal representation. Consequently, the formal logic of program correctness does nothing for user concerns about performance, since they abstract away memory layout details which is crucial for understanding the impact of hardware factors such as bus latency and bandwidth.

As a result of the more liberal attitude taken toward program validation, dynamic languages lend naturally to rapid iteration through many prototypes and versions of computer programs. This sort of experimentation in writing programs comes naturally to technical computing users, who often have to write programs without any idea of what the final result ought to look like. Dynamic languages offer a natural expression for these use cases that lack formal specifications, by not imposing upon users the burden of writing formally correct programs, allowing them to focus instead on expressing how to do practical computations.

\paragraph{Multiple dispatch allows dispatch on new types and new functions at the same time}

%TODO This is very garbled and mostly reflects my lack of understanding of dispatch systems - cjh

The needs of technical computing can exceed the abilities of most dispatch systems. In particular, it is usually not possible to have new behaviors that intermix new types and new functions. One group of languages allows you to have new types but their dispatch upon existing functions cannot be defined. Subtyping is the usual paradigm here, but it is closed because it assumes you've covered all the cases explicitly. Object oriented programming can be seen as a solution around this problem, but in pure OO, objects have only identity and have no interface protocol. Classes are a mechanism for implementing message sends, defining actions \textit{upon} an object. The other group of languages allows you to have new functions but not for existing types. Haskell typeclasses~\cite{typeclass} are a fixed collection of interfaces; while additional functions can be defined, only the functions that form part of the existing interface can interact with an existing type.

\paragraph{Fewer language constructs for user simplicity}

The original motivation in Julia for having types as values was to simplify the
language for technical computing users. The current design came about from
considering the desire for users to simplify code dealing with parametrically
polymorphic types, i.e. be able to write things like \code{Array} as a synonym
for \code{Array\{T\}}. Such synonyms are valuable for composability when writing 
methods that are agnostic about the type parameter \code{T}. Methods that cared
about \code{T} can be defined with a type signature that explicitly mentions
\code{T} and methods that did not care about \code{T} can have a type signature
that left it out.

%XXX like R? I forgot which language Jeff mentioned
The ability to leave out type parameters in function signatures contrasts with
other languages which require explicit specification of all type parameters,
resulting in users having to write redundant code blocks whose sole purpose is
handle the nuisance parameter. In contrast, the value proposition in Julia is
that having types as values, a collapsed kind hierarchy (i.e. making no
distinction between types and kinds (meta-types) and meta-kinds etc.), plus
being able to reason about types dynamically, affords users a way to harmonize
the use of types and pattern matching language constructions that are common in
other languages.


\section{Introductory Example}

Code for technical computing often sacrifices abstraction for performance,
and is less expressive as a result. In contrast, mathematical ideas are
inherently polymorphic and amenable to abstraction. Consider a simple example
like multiplication, represented in Julia by the \lstinline|*| operator.
\lstinline|*| is a generic function in Julia, and has specialized methods for
many different multiplications, such as scalar--scalar products, scalar--vector
products, and matrix--vector products. Expressing all these operations using
the same generic function captures the common metaphor of multiplication.

\subsection{Bilinear forms}

Julia allows user code to extend the \lstinline|*| operator, which can be
useful for more specialized products. One such example is bilinear forms, which
are vector--matrix--vector products of the form
%
\begin{equation}
\gamma = v^\prime M w = \sum_{ij} v_i M_{ij} w_j
\end{equation}
%
This bilinear form can be expressed in Julia code of the form
%
\begin{lstlisting}
function (*){T}(v::AbstractVector{T}, 
                M::AbstractMatrix{T},
                w::AbstractVector{T})
    if !(size(M,1) == length(v) && 
         size(M, 2) == length(w))
        throw(BoundsError())
    end
    $\gamma$ = zero(T)
    for i = 1:size(M,1), j = 1:size(M,2)
        $\gamma$ += v[i] * M[i,j] * w[j]
    end
    return $\gamma$
end
\end{lstlisting}
%
The newly defined method can be called in an expression like
\lstinline|v * M * w|, which is parsed and desugared into an ordinary function
call \lstinline|*(v, M, w)|. This method takes advantage of the result being a
scalar to avoid allocating intermediate vector quantities, which would be
necessary if the products were evaluated pairwise like in $v^\prime(Mw)$ and
$(v^\prime M) w$. Avoiding memory allocation reduces the number of
heap-allocated objects and produces less garbage, both of which are important
for performance considerations.

The method signature above demonstrates two kinds of polymorphism in Julia.
First, both \lstinline|AbstractVector| and \lstinline|AbstractMatrix| are
abstract types, which are declared supertypes of concrete types. Examples of
subtypes of \lstinline|AbstractMatrix| include \lstinline|Matrix| (dense
two-dimensional arrays) and \lstinline|SparseMatrixCSC| (sparse matrices stored
in so-called compressed sparse column format). Thus the method above is defined
equally for arrays of the appropriate ranks, be they dense, sparse, or even
distributed.  Second, \lstinline|T| defines a type parameter that is common to
\lstinline|v|, \lstinline|M| and \lstinline|w|. In this instance, \lstinline|T|
describes the type of element stored in the \lstinline|AbstractVector| or
\lstinline|AbstractMatrix|, and the \lstinline|{T}(...{T}...{T})| syntax
defines a family of methods where \lstinline|T| is the same for all three
arguments. The type parameter \lstinline|T| can also used in the function body;
here, it is used to initialize a zero value of the appropriate type for
$\gamma$.

The initialization statement \lstinline|$\gamma$ = zero(T)| allows Julia's compiler
to generate \textbf{type stable} code. If \lstinline|T| is a concrete immutable
type, e.g.\ \lstinline|Float64| (64-bit floating point real numbers), then
Julia's just-in-time compiler can analyze the code statically to remove type
checks.  For example, in the method with signature
\lstinline|*{Float64}(v::AbstractVector{Float64}, M::AbstractMatrix{Float64}, w::AbstractVector{Float64})|,
the indexing operations on $v$, $M$ and $w$ always return \lstinline|Float64|
scalars. Furthermore, forward data flow analysis allows the type of $\gamma$
to be inferred as \lstinline|Float64| also, since floating point numbers are
closed under addition and multiplication. Hence, type checks and method
dispatch for functions like \lstinline|+| and \lstinline|*| within the function
body can be resolved statically and eliminated from run time code, allowing
fast code to be generated.

Were we to replace the initialization with the similar-looking
\lstinline|$\gamma$ = 0|, we would have instead a \textbf{type instability}
when \lstinline|T = Float64|. Because $\gamma$ is initialized to an
\lstinline|Int| (native machine integer) and it is incremented zero or more
times by a \lstinline|Float64| in the \lstinline|for| loop, the compiler cannot
determine a concrete type for $\gamma$ at compile time, since the actual
concrete type depends on the size of the input array $M$, which can only be
determined by the specific value bound to $M$. Instead, $\gamma$ is inferred to
be the type union \lstinline|Union(Int,Float64)|, which is the least upper
bound on the actual type of $\gamma$ at run time. As a result, not all the type
checks and method dispatches can be hoisted out of the function body, resulting
in slower run time code.



\subsection{Matrix equivalences}

Matrix equivalences are another example of specialized product that users may
want. Two $n\times n$ matrices $A$ and $B$ are considered equivalent if there
exist invertible matrices $V$ and $W$ such that $B = V * A * W^\prime$.
Oftentimes, equivalence relations are considered between a given matrix $B$ and
another matrix $A$ with special structure, and the transformation
$(W^\prime)^{-1} \cdot V^{-1}$ can be thought of as changing the bases of the
rows and columns of $B$ to uncover the special structure buried within as $A$.
Matrices with special structure are ubiquitous in numerical linear algebra. One
example is rank-$k$ matrices, which can be written in the outer product form $A
= X Y^\prime$ where $X$ and $Y$ each have $k$ columns. Rank-$k$ matrices may be
reified as dense two-dimensional arrays, but the matrix--matrix--matrix product
$V * A * W^\prime$ would take $O(n^3)$ time to compute. Instead, when $k \ll
n$, the product be computed more efficiently in $O(kn^2)$ time, since
%
\begin{equation}
V A W^\prime = V (X Y^\prime) W^\prime = (V X) (W Y)^\prime
\end{equation}
%
and the result is again a rank-$k$ matrix. Furthermore, we can avoid
constructing $W^\prime$, the transpose of $W$, explicitly. Therefore in some
cases, it is sensible to store $A$ as the two matrices $X$ and $Y$ separately,
rather than as a reified 2D array.

Julia allows users to encapsulate $X$ and $Y$ within a specialized type:
%
\begin{lstlisting}
type OuterProduct{T}
	X :: Matrix{T}
	Y :: Matrix{T}
end
\end{lstlisting}
%
Defining the new \lstinline|OuterProduct| type has the advantage of grouping
together objects that belong together, but also enables dispatch based on the
new type itself. We can now write a new method for \lstinline|*|:
%
\begin{lstlisting}
*(V, M::OuterProduct, W) = OuterProduct(V*M.X, W*M.Y)
\end{lstlisting}
%
This method definition uses a convenient one-line syntax for short definitions
instead of the \lstinline|function ... end| block. This method also does not
annotate \lstinline|V| or \lstinline|W| with types, and so they are considered to
be of the top type $\top$ (\lstinline|Any| in Julia). This method may be called
with any $V$ and $W$ which support premultiplication: so long as
\lstinline|V*M.X| and \lstinline|W*M.Y| are defined and produce matrices of the
same type, then the code will run without errors. This flexibility is
convenient since $V$ and $W$ can now be scalars or matrixlike objects which
themselves have special structures, or even more generally could represent
linear maps that are not stored explicitly, but rather defined implicitly
through their actions when multiplying a \lstinline|Matrix| on the left.

The preceding method shows that Julia does not require all method arguments to
have explicit type annotations. Instead, dynamic multiple dispatch allows the
argument types to be determined from the arguments at run time. Nevertheless,
Julia's just-in-time compiler can still be invoked when the method is first
called, so that static analyses, such as type inference based on forward data
flow, and optimizations like function inlining, can be performed. Late binding
dynamic dispatch thus give us the ability to write highly generic code while
still allowing for aggessive method specialization on methods that are actually
dispatched upon at run time.

Furthermore, late binding allows for methods to be defined even on types which
have not yet been defined. For example, we can now proceed to define a new type
and method

\begin{lstlisting}
type RowPermutation
	p::Vector{Int}
end

*($\Pi$::RowPermutation, M::Matrix) = M[p, :]
\end{lstlisting}
%
whose action can be thought of as multiplying by a permutation matrix on the
left, resulting in a version of $M$ with the rows permuted. Now, the following
user code
%
\begin{lstlisting}
n = 10
k = 2
X = randn(n, k) #Random matrix of Float64s
M = OuterProduct(X, X)
p = randperm(n) #Random permutation of length n
$\Pi$ = RowPermutation(p)
M2 = $\Pi$ * M * $\Pi$
\end{lstlisting}
%
will dispatch on the appropriate methods of \lstinline|*| to produce the same
result \lstinline|M2| as
\begin{lstlisting}
M2 = OuterProduct(M.X[p, :], M.Y[p, :])
\end{lstlisting}
%
In other words, the specialized method
\lstinline|*(::RowPermutation, ::OuterProduct{Float64}, ::RowPermutation)|
is compiled only when it is first invoked in the creation of
\lstinline|M2|, since it follows from composing the method defined with
signature \lstinline|*(::Any, ::OuterProduct, ::Any)| with the argument tuple
of type \lstinline|(RowPermutation, OuterProduct{Float64}, RowPermutation)|.



\subsection{Late binding increases expressiveness}

The examples in this section, while simple, illustrate the expressive power
afforded by the composition of extensible generic functions, polymorphic types,
dynamic multiple dispatch, and aggressive method specialization allowed by
just-in-time static analyses. Users are allowed to extend both the collection
of generic functions and the base type hierarchy in Julia, which further erodes
the distinction between user code and library code that is itself written in
Julia. We believe that such expressivity is useful for technical computing
applications, where it is not generally possible to predict the variety of
specialized computations that domain scientists, engineeers and mathematicians
require.

Some other languages also offer constructs for type polymorphism, like C++'s
expression templates and Fortress's generic functions, but these constructs are
only available at compile time, which restrict their expressiveness.
Furthermore, these languages usually require that all possible specialized
methods be generated at compile time, resulting in long compilation times which
are curtailed in practice with further restrictions on the generality of user
defined code. In contrast, the combination of dynamic multiple dispatch
semantics and on-demand method specialization in Julia allows users to write
highly generic code. Consider that the method definitions above each represent
an infinite family of methods. While user code in practice only uses a small,
finite subset of the possible methods, users have the luxury of choosing from
the entire universe encompassed by the generic function system.


\section{The Julia Language}
\TODO{Give an overview of the Julia language.}

\section{The Julia Type System}

\begin{quote}
  \textit{Ceci n'est pas une type} \\
  (With apologies to H. Magritte)
\end{quote}

We now define the types in Julia. We have designed the type system to be as permissive as possible to programmers while still providing enough information to statically resolve a significant portion of dispatches. The main contribution of the type system design is that types are values that the programmer can write programs to compute. The challenge in designing such a type system are in specifying how tags can be computed and how the subtyping relationships work. [Question: is this type system based on/similar to any other type system?]
Motivating reason: having types as values allows you to have fewer components to the language design. Lets you write array parametric on a type without explicitly declaring the type parameter.
Main challenges:
- Types/tags exist to describe things in a standard way. Challenge to think about what needs to be described. Canonical example for long time was arrays with different element types.
[What were the main difficulties in designing the type system?]
[What about promoters? Do type promotions show up in other languages?]
Type promotions are written as user-level code to compute new types.

In Figure [FIGURE REF] we show Julia’s types. A type can be an abstract [what the heck is an abstract], a data constructor [is this a record?], a tuple, a quantification over a set of types, a union, or a singleton type. In the code, singletons are used for computations on types themselves, such as type promotions. For example, given two types, can ask what the promoted type is. So far haven’t had singleton types for general values. (Sorting algorithms example: used more for dispatch as shorthand for type with no parameters.) Will use this more because generating specialized code is big for scientific computing, but need to figure out what to specialize on and how to tell system what to specialize on. State of the art is to have ad hoc systems.

An abstract type declares a type without a representation. It takes the name of a type, parameters [what do the parameters do?], and the name of the super type. Data type declarations additionally provide a representation. [TODO: Talk about the representation.]
Abstract type and data type declarations are invariant and nominative.

Tuples are covariant.

ForAll types quantify over all types between a lower bound and upper bound. Have lower bound and upper bound because of containers that can be read and written. People mostly only use the upper bounds. No function types, so no contravariance, so lower bounds don’t get used much. In theory, can still use them for functions that mutate things.

[How does type promotion fit into all this?]

\begin{minted}[frame=lines,framesep=2mm]{julia}
Type ::= Abstract | Data | Tuple | ForAll | Union | Singleton
Abstract ::= Name (P) Super }
Data ::= Name (P) Super Repr } invariant, nominative
TODO: What is repr?
Tuple ::= (T1, T2, …) | (T1, T2, …, Tn, …) } covariant
ForAll ::= $\forall$ (lb <: T <: ub) . Type
Union ::= U (T1, T2, …)
Singleton ::= Lift Value
TypeVar ::= lb <: Name <: ub
top ::= Any
--
Tag ::= Data | Tuple
\end{minted}

\subsection{Basics}
\TODO{This is where we write one of those trees.}

Can write procedures to subtype.

Jeff is pretty sure subtyping is decidable and well-defined
Join is union of any two types; meet is less well-defined in this lattice

What are types in Julia?
Dynamically typed.
What is inheritance like?
How does Julia avoid problems that OO languages tend to have with dispatch? (I’m not actually sure what they are.)
Is there multiple inheritance?
No, but it’s been discussed (https://github.com/JuliaLang/julia/issues/5)
Is there multiple subtyping?
No, each type has only one supertype
We should also describe the dispatch mechanism.
Method tables are a sorted list of these types.
Implemented as \code{jl\_methtable\_t}

\paragraph{What is a type?}

The defining feature of dynamic typing is that all values, semantically at least, have
two parts: a \emph{tag} and some \emph{data}. The tag classifies the value according to
some ontology defined by the programming language, and the data is a block of memory
whose format is set by the programming language, possibly in a way that depends on the
tag.

Colloquially tags are typically called ``types'' by programmers, as the distinction
is not important in most uses of dynamic typing. In type theory each tag corresponds
one-to-one to a type encoding the proposition that some term evaluates to a value
with that tag. However dynamically-typed languages usually do not require that such
a type system be used. Statically determining all tags is generally not possible.
Furthermore, it would be perfectly reasonable to impose a static type system that
was not concerned with tags at all, but rather with other program properties
(e.g. checking for possible uses of null references). Tags are a mechanism, and as
such neither require nor preclude any particular formalism.

Tags are valuable because they have a constrained structure. While the data
part of a value may vary arbitrarily at run time, tags are drawn from a limited,
well-understood family. This provides for self-describing data: it becomes possible
to write a program that accepts an \emph{arbitrary} value, discovers its structure,
and operates on it, by examining its tag.

The potential of tags is to provide a common descriptive language shared by all users
of a language, as well as the compiler. The cost of tags is overhead. The addition of
a tag may double, or more, the memory footprint of a data item. Partly for this reason,
most dynamically typed languages try to simplify and minimize their tag systems.
(Laurence Tratt also points out that tags tend to resemble the types used in static
languages, likely as a result of cultural expectations \cite{}.) Taken to an extreme,
as in Scheme, the set of possible tags might be finite and small, allowing tags to be
overlaid with pointer bits in many implementations.

Julia tries to reach the other extreme, providing for tags with nested structure, and
possibly containing arbitrary values.

- describing when code is applicable (dispatch)
- describing what to specialize on
- describing memory layout
- describing what, if anything, is statically known about a potential value

\paragraph{Dataflow analysis}

What is the goal of the dataflow analysis? Primarily for static type inference. Eliminate runtime checks. Lattice-based. Kaplan-Ullman.
Problem is finding the tightest possible set of allowable types.
There are tweaks to the standard algorithm. One big thing is that dataflow analysis only goes in forward direction.
Kaplan-Ullman also doesn’t treat parametric types
Julia also has a widening step.
Relies on types having only one supertype.
Diagonal dispatch: can constrain arguments to same type.

\TODO{What are the challenges?}

\TODO{What is the complexity?}

\TODO{What is the best way to express this?}

\TODO{bit of background}

\paragraph{Type inference is a key part of the language}
A somewhat unusual feature of julia is that we consider dataflow type inference a
key part of the language. Strictly speaking, this is an optional, external program
analysis that might be used for various purposes, chief among them implementing
an optimizing compiler. However, it is highly important to programmers since it
largely defines not the semantics, but the performance model of the language.

We feel that dataflow analysis, especially of forward flow, captures a piece of
the human intuition of how programs work: values start at the top and move through
the program step by step. For example, compilers are much more user-friendly
when they elide a possibly-uninitialized variable warning in

\begin{minted}[frame=lines,framesep=2mm]{julia}
int a;
if (cond)
    a = 1;
else
    a = 2;
f(a);
\end{minted}

The programmer knows that \code{a} is always initialized before use.

\paragraph{Subtyping relations and the type lattice}

\paragraph{\code{typeof}}

You can define \code{typeof} axiomatically as having the following behavior on a \code{value=(bits, tag)} pair:

\begin{minted}[frame=lines,framesep=2mm]{julia}
typeof(bits, tag) = (tag, DataType) #Returns a value
\end{minted}

\code{typeof} has a fixed point, namely \code{(DataType, DataType)}. This is also true in other dynamic languages, e.g. Python (CPython). In other languages like Haskell, \code{typeof(DataType) = Kind}, etc. Static languages can just truncate the tower of metatypes and also refuse to type-check code that reasons about types and kinds too far up the hierarchy. In fact, early versions of Haskell did not allow for programs to reason about kinds at the data type level due to the lack of kind polymorphism~\cite{haskellkindtypes}.

\TODO{There is a subtlety about typeof's behavior. typeof is a projection; typeof(not-a-type) produces a DataType, which projects non-type values onto types. It also has the effect of lifting non-type values onto a type lattice; the latter is defined only for values that are DataTypes.}

\paragraph{Widening}

\TODO{Formal proof of correctness?}

\subsection{Dispatch}

\paragraph{method sorting for specificity}

\TODO{The main novelty and challenge is explaining type parameters and typevars. Currently typevars are not first-class objects in Julia; you can't pass them to a function. Expressions of the form \code{T<:SomeType} don't have an independent existence outside a function signature that also contains the \code{\{T...\}} construction.}

\subsection{Type Inference}

\subsection{Type promotion}

\subsection{Example: modular integer arithmetic with \code{lcm}}

Here's an example of a type parameter computed with the \code{lcm} function:

\begin{minted}[frame=lines,fontsize=\footnotesize,
               framesep=2mm]{julia}
import Base: convert, promote_rule, show, showcompact

immutable ModInt{n} <: Integer
    k::Int
    ModInt(k) = new(mod(k,n))
end

-{n}(a::ModInt{n}) = ModInt{n}(-a.k)
+{n}(a::ModInt{n}, b::ModInt{n}) = ModInt{n}(a.k+b.k)
-{n}(a::ModInt{n}, b::ModInt{n}) = ModInt{n}(a.k-b.k)
*{n}(a::ModInt{n}, b::ModInt{n}) = ModInt{n}(a.k*b.k)

convert{n}(::Type{ModInt{n}}, k::Int) = ModInt{n}(k)
convert{n}(::Type{ModInt{n}}, k::ModInt) = ModInt{n}(k.k)
promote_rule{n}(::Type{ModInt{n}}, ::Type{Int}) = ModInt{n}
promote_rule{m,n}(::Type{ModInt{m}}, ::Type{ModInt{n}}) =
    ModInt{lcm(m,n)}

show{n}(io::IO, k::ModInt{n}) = print(io, "$(k.k) mod $n")
showcompact(io::IO, k::ModInt) = print(io, k.k)

julia> a = ModInt{12}(18278176231)
7 mod 12

julia> b = ModInt{15}(2837628736423)
13 mod 15

julia> a + b
20 mod 60
\end{minted}

The type of the result \code{a + b} depends on the types of \code{a} and \code{b} via \code{lcm}.

\section{Performance}

\TODO{Talk about how much type inference improves.}

\TODO{Talk about gains over non-Julia languages.}

\section{Dispatch in Practice}
\TODO{Talk about the library examples.}

\section{Discussion}
\TODO{Talk about the psychological implications of types.}

\section{Related Work}

There has been a rich history in using JIT compiler techniques to improve the performance of dynamic languages used for technical computing. 
Matlab has had a production JIT compiler since 2002~\cite{matlab2002matlab}.
More recently LuaJIT~\cite{pall2008luajit} and PyPy~\cite{Bolz2009} have shown that sophisticated tracing JIT's can significantly improve the runtime performance of dynamic languages.
Julia's compiler takes advantage of LLVM's JIT for performance, but effort has been directed toward language design and not on improving existing JIT compiler techniques or implementations.
Multimethods, polymorphic types, and multiple dispatch are exemplar language features that allow for greater opportunity for dynamic code optimization.

Multiple dispatch using through multimethods has been explored in a variety of programming languages, either as a built in construct or as a library extension.
A limited sampling of programming languages that support dynamic multiple dispatch are Cecil~\cite{Chambers1992,Chambers1994}, Common Lisp's CLOS~\cite{Bobrow1988}, Dylan~\cite{dylanman}, Clojure~\cite{Hickey2008}, and Fortress~\cite{Allen2011}.
These languages differ in their dispatch rules.  
Cecil, and Fortress employ symmetric multiple dispatch similar to Julia's dispatch semantics.
Common lisp's CLOS and Dylan generic functions rely on asymmetric multiple dispatch, resoling ambiguities in method selection my matching arguments from left to right.
Multimethods not part of the core method system or as a user level library can have user defined method selection semantics.
Such a system is implemented in Clojure~\cite{Hickey2008}, which can reflect on the runtime values of a method's arguments, not just its types, when doing method selection.

Method dispatch in these languages is limited by the expressiveness of the their type systems.
Clojure's multimethods are not a core language construct and only weakly interact with built-in types.
Dispatch in CLOS is class-based and excludes parametric types and cannot dispatch off of Common lisp's value types, limiting its applicability as a mechanism for optimized method selection.  
Cecil's type system supports subtype polymorphism but not type parameters.
Dylan supports CLOS-style class-based dispatch, and also let-polymorphism in limited types, which is a restricted form of parametric polymorphism.
However, Dylan does allows for multiple inheritance. 

Julia is most similar to Fortress in exploring the design space of multiple dispatch with polymorphic types as a mechanism for supporting static analysis. Fortress has additional complexity in its type system, allowing for multiple inheritance, traits, and method selection forcing mechanisms.  This is in contrast to Julia's simpler polymorphic type hierarchy which enforces single inheritance. Static analysis in Fortress is mostly limited to checking method applicability for type correctness.

%% Different than other multiple dipatch languages where we focus on using multiple dispatch as a mechnaism for static analysis,
%% Fortress also does static analysis but the focus on the static analysis was for checking method applicibility for type correctness.

%% Common lisp has a notion of primitve types in the where you can write that an array is arary{int} but you cannot dispatch based on that fact

%% C# the dynamic keyword allows for method selection based on the runtime argument types but the method selection rules are complex
%% Groovy allows for dynamic dispatch but but I'm unsure whether it allows for true multiple dispatch,
%% All the given examples show mostly method overloading and runtime method selection based on argument class reflection

\section{Conclusions}


\acks{We thank the many Julia users and developers for their contributions to
the Julia language.}

\listoftodos %TODO remove from final submission

%\tableofcontents

% We recommend abbrvnat bibliography style.

\bibliographystyle{abbrvnat}
\bibliography{pldi2015,websites}

\end{document}
