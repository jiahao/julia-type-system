\paragraph{Exotypes in Lua}
Exotypes\cite{exotypes} provide to the Lua language\cite{lua} the analogue of macros that generate staged functions\cite{stagedfunc} in Julia.

There are some notable differences, though:

Exotypes are programmed in Terra,\cite{terra} a separate layer on top of the Lua host language, and method specialization must be explicitly invoked from within Terra. The syntax incurred by having two language layers creates an artificial distinction between builtin methods and user-defined methods that doesn't exist in Julia.

Exotypes implement automatic broadcasting when there is a \code{\_\_methodmissing} property defined. Julia does not use automatic broadcasting to implement the Proxy design pattern.

Julia uses an eager approach to resolving circularity in method definitions. Exotypes use lazy evaluation, although interestingly, earlier versions of Terra also used eager evaluation~\cite{terra}.

\TODO{Multiple dispatch paradigm may obviate the need for exotypes in some uses and make exotypes more powerful in others}

In a multiple dispatch paradigm, the issues brought up in Lua/Terra in wanting to being abole to create types with an unbounded number of behaviors becomes irrelvant. There is a separation of functions and objects in Julia that obviates the need for lazily queried properties (although it may still be a more efficient implementation choice).