\begin{abstract}
The goal of scientific programs is often to create experiments rather than to build robust systems. Because scientific programs frequently involve operating over reified data values with difficult-to-predict types, it is often easier for scientific programmers to use dynamically typed languages. Unfortunately, dynamic types often make it difficult to execute code efficiently. In this paper, we describe the type system and multiple dispatch mechanism for the Julia programming language, specialized for scientific computing. Julia has a novel semantics based on efficient multiple dispatch. Julia combines programmer-specified type tags with a type inference engine to statically optimize method dispatch. Julia is dynamically typed in that the compiler does not statically reject ill-typed programs, but Julia is designed to optimize static type and dispatch inference. We describe the Julia language and type system and our implementation of type inference and multiple dispatch mechanisms for Julia. We demonstrate [X]x performance improvements over [MATLAB? un-optimized Julia code?]. To demonstrate the relevance and potential impact of designing a language based around multiple dispatch, we also describe the the manifestations of multiple dispatch in Julia's standard library.
\end{abstract}
